\documentclass[12pt]{article}

%----------------------Autor y fecha----------------------

\title{Matemática Superior}
\author{\bfseries{Trabajo Práctico 1}\\
Clara Mazzi, Miguel Storani}
\date{26 de mayo de 2019i}


%------------------------Paquetes-------------------------

\usepackage{graphicx}			% Gráficos
\usepackage[utf8]{inputenc}		% Codificación
\usepackage[spanish]{babel}		% Idioma
\usepackage{lastpage}
\usepackage{dcolumn}			% Tablas
\usepackage{enumitem}			% Listas
\usepackage{bm}					% Negrita en ecuaciones
%\usepackage{pstricks}			% Arboles
%\usepackage{pst-tree}			% 
\usepackage{amsmath}
\usepackage{amssymb}
%---------------------------------------------------------

%------------------------Márgenes-------------------------

\usepackage{vmargin}
\setpapersize{A4}
\setmargins{3cm}	% margen izquierdo
{1.25cm}			% margen superior
{16.5cm}			% anchura del texto
{24.2cm}			% altura del texto
{18pt}				% altura de los encabezados
{1cm}				% espacio entre el texto y los encabezados
{10pt}				% altura del pie de página
{2cm}				% espacio entre el texto y el pie de página
%---------------------------------------------------------

%-----------Tamaño de Encabezado y Pie de pagina----------

\usepackage{fancyhdr}						% Paquete

\setlength{\headwidth}{\textwidth}
\setlength{\headheight}{28pt}
\setlength{\footskip}{1cm}
\renewcommand{\headrulewidth}{0.4pt}
\renewcommand{\footrulewidth}{1pt}
%---------------------------------------------------------

%---------------------Diseño Encabezado-------------------

\fancyhead[L]{\includegraphics[height=.5cm]{./logos/UTN-FRSF.jpg}}
\fancyhead[R]{\textsf{\Large{Ingeniería en Sistemas de Informacíon}}}
%---------------------------------------------------------

%--------------------Diseño Pié de pagina-----------------


\fancyfoot[R] {\thepage \textnormal{ de} \pageref{LastPage}}
\fancyfoot[L]{\textsc{Storani}, Miguel Ignacio}
\fancyfoot[C]{2019}
%---------------------------------------------------------



%\renewcommand{\familydefault}{\sfdefault}	% Cambio de fuente por defecto

\pagestyle{empty}							% Estilo de página (Habilita los encabezados y pié de página)




\begin{document}

%----------------------Carátula---------------------------
\pagestyle{empty}

\begin{titlepage}

\newcommand{\HRule}{\rule{\linewidth}{0.5mm}} % Defines a new command for the horizontal lines, change thickness here

\begin{center} % Center everything on the page
 
%----------------------------------------------------------------------------------------
%	HEADING SECTIONS
%----------------------------------------------------------------------------------------

\textsc{\LARGE Universidad Tecnológica Nacional}\\[0.5cm]				% Name of your university/college
\textsc{\Large Facultad Regional Santa Fe}\\[1.5cm]						% Major heading such as course name
\textsc{\large Ingeniería en Sistemas de Información}\\[0.5cm]			% Minor heading such as course title

%----------------------------------------------------------------------------------------
%	TITLE SECTION
%----------------------------------------------------------------------------------------

\HRule \\[0.9cm]
{ \huge \bfseries Matemática Superior}\\[0.4cm]
{ \large \bfseries Trabajo Práctico 1}\\[0.7cm] % Title of your document
\HRule \\[1.5cm]

%----------------------------------------------------------------------------------------
%	AUTHOR SECTION
%----------------------------------------------------------------------------------------

\begin{minipage}{0.8\textwidth}
\begin{flushleft}\large
\emph{Alumnos:} \\
\hspace{0.5cm}  \textsc{Mazzi}, Maria Clara
{\small \texttt{mazzimclara@gmail.com}}\\
\hspace{0.5cm}  \textsc{Storani}, Miguel Ignacio
{\small \texttt{miguelignaciostorani@gmail.com}}\\



\end{flushleft}
\end{minipage}\\[2.5cm]


% If you don't want a supervisor, uncomment the two lines below and remove the section above
%\Large \emph{Author:}\\
%John \textsc{Smith}\\[3cm] % Your name

%-----------------------------------------------------------------------
%	DATE SECTION
%-----------------------------------------------------------------------



%-----------------------------------------------------------------------
%	LOGO SECTION
%-----------------------------------------------------------------------

\includegraphics[width=5cm]{./logos/logo_utn.png}\\[1cm] 				% Include a department/university logo - this will require the graphicx package
 {\large 26 de mayo de 2019}											% Date, change the \today to a set date if you want to be precise
%-----------------------------------------------------------------------

\newpage % Fill the rest of the page with whitespace
\end{center}
%%\end{titlepage}
%%---------------------------------------------------------

\begin{abstract}
kjdfhgjhsdfjkvbnsjkfdvbkjfdbv
\end{abstract}

%--------------------------Índice-------------------------

\tableofcontents\vspace{2.5cm}
\listoffigures
\thispagestyle{empty}
\end{titlepage}
\newpage
%---------------------------------------------------------
\pagestyle{fancy}
%############################
%## Acá escribí tu trabajo ##
%############################



\section{Expresar la nueva transformada en función de transformadas conocidas}

\begin{equation}
\overline{T}_{ms}(f_{(t)}) = \frac12
\begin{pmatrix}
\int_{-\infty}^{\infty} f_{(t)} e^{-z |t|}dt\\
\int_{-\infty}^{\infty} f_{(t)} \textrm{sign}(t)e^{-z |t|}dt
\end{pmatrix}
\label{transformada_original}
\end{equation}

Para definir a nuestra transformada en función de tranformadas conocidas operamos separadamente con cada una de las componentes que la componen.
$$
\int\limits_{-\infty}^{\infty} f_{(t)} e^{-z |t|}dt= \int\limits_{0}^{\infty} f_{(t)} e^{-z t}dt + \int\limits_{-\infty}^{0} f_{(t)} e^{-z (-t)}dt$$
\begin{equation}
\int\limits_{-\infty}^{\infty} f_{(t)} e^{-z |t|}dt= \mathcal{L}\left\{f(t)u(t)\right\}_{(s)} +  \int\limits_{-\infty}^{0} f_{(t)} e^{-z (-t)}dt
\label{primer_componente}
\end{equation}

Resolviendo separadamente el segundo término de la expresión, aplicando un cambio de variable tal que $t=-v$ tenemos que:

$$ \int\limits_{-\infty}^{0} f(t) e^{-z (-t)}dt \Rightarrow - \int\limits_{\infty}^{0} f(-v) e^{-z (v)}dv =  \int\limits_{0}^{\infty} f(-v) e^{-z (v)}dv $$
$$\int\limits_{0}^{\infty} f(-v) e^{-z (v)}dv = \mathcal{L}\left\{f(-v) u(v)\right\}_{(z)}$$

Por propiedad de simetría de la transformada de Laplace, se puede expresar la siguiente igualdad:
 \begin{equation}
 \int\limits_{-\infty}^{0} f(t) e^{-z (-t)}dt = \mathcal{L}\left\{f(t)u(-t)\right\}_{(-z)} 
 \label{segundo_termino}
 \end{equation}
 
Lo cual representa un cambio de la variable compleja, por la misma variable compleja aplicando un factor de $-1$. Reemplazando \ref{segundo_termino} en \ref{primer_componente} obtenemos que:

\begin{equation}
\int\limits_{-\infty}^{\infty} f_{(t)} e^{-z |t|}dt= \mathcal{L}\left\{f(t)u(t)\right\}_{(z)} +  \mathcal{L}\left\{f(t)u(-t)\right\}_{(-z)}
\label{primer_componente_final}
\end{equation}

Pasamos ahora a operar con la segunda componente de nuestra transformada.

$$\int\limits_{-\infty}^{\infty} f_{(t)} \textrm{sign}(t)e^{-z |t|}dt$$

Podemos definir a la función sign$(t)$ en función del escalón unitario, quedando sign$(t) = u(t)-u(-t)$, por lo tanto:
$$\int\limits_{-\infty}^{\infty} f_{(t)} \textrm{sign}(t)e^{-z |t|}dt = \int\limits_{-\infty}^{\infty} f_{(t)} (u(t) -u(-t))e^{-z |t|}dt$$

$$\int\limits_{-\infty}^{\infty} f_{(t)} (u(t) -u(-t))e^{-z |t|}dt =  \int\limits_{-\infty}^{\infty} f(t)u(t) e^{-z |t|}dt - \int\limits_{-\infty}^{\infty} f_{(t)}u(-t) e^{-z |t|}dt$$

\begin{equation}
\int\limits_{-\infty}^{\infty} f_{(t)} (u(t) -u(-t))e^{-z |t|}dt =   \mathcal{L}\left\{f(t)u(t)\right\}_{(s)} - \int\limits_{-\infty}^{\infty} f_{(t)}u(-t) e^{-z |t|}dt
\label{transformada_a_la_mitad}
\end{equation}

El segundo término se puede expresar como:
\begin{equation} \int\limits_{-\infty}^{\infty} f_{(t)}u(-t) e^{-z |t|}dt = \int\limits_{-\infty}^{0} f(t) e^{-z (-t)}dt \Rightarrow  \mathcal{L}\left\{f(t)u(-t)\right\}_{(-z)}
\label{transformada_segundo_termino}
\end{equation}

Expresión para la que ya hemos hayado una expresión en función de transformadas conocidas, por lo que la segunda componente de nuestra tranformada quedaría de reemplazar \ref{transformada_segundo_termino} en \ref{transformada_a_la_mitad}:
 \begin{equation}
 \int\limits_{-\infty}^{\infty} f_{(t)} \textrm{sign}(t)e^{-z |t|}dt =    \mathcal{L}\left\{f(t)u(t)\right\}_{(z)} -  \mathcal{L}\left\{f(t)u(-t)\right\}_{(-z)}
 \label{segunda_componente_final}
 \end{equation}


Luego de haber calculado ambas componentes, reemplazamos \ref{primer_componente_final} y \ref{segunda_componente_final} en \ref{transformada_original} nos queda que nuestra transformada se puede expresar como:

\begin{equation}
\overline{T}_{ms}(f_{(t)}) = \frac12
\begin{pmatrix}
\mathcal{L}\left\{f(t)u(t)\right\}_{(z)} +  \mathcal{L}\left\{f(t)u(-t)\right\}_{(-z)}\\[0.2 cm]
\mathcal{L}\left\{f(t)u(t)\right\}_{(z)} -  \mathcal{L}\left\{f(t)u(-t)\right\}_{(-z)} 
\end{pmatrix} =  \frac12
\begin{pmatrix}
\mathcal{L}_u\left\{f(t) + f(-t)\right\}\\[0.2 cm]
\mathcal{L}_u\left\{f(t) -f(-t)\right\}
\end{pmatrix}
\label{transformada_con_laplace}
\end{equation}

\section{Condiciones necesarias y suficientes para garantizar la existencia de la tranformada}

Para asegurar la existencia de la transformada se debe cumplir que exista la transformada de Laplace para la función estudiada. Existe la transformada $\overline{T}_{ms}(f_{(t)})$ siempre que $f_{(t)}$ sea una función que cumple que:

\begin{itemize}
\item Es seccionalmente continua sobre el intervalo $t \le A$ para cualquier $A > 0$, esto es, posee a lo más un número finito de discontinuidades de salto en dicho intervalo.
\item Es de orden exponencial para $t \ge M$, es decir:
\center{ $|f(t)| \le Ke^{at}$ para $t\ge M$ donde $K$, $a$ y$M$ son constantes}
\end{itemize}


\section{Antitransformada}

Para reconstruir la función original a partir de la transformada, tomamos las componentes en $x$ e $y$ de la transformada por separado, siendo:

$$\overline{T}_{ms}(f_{(t)}) =
\begin{pmatrix}
T_x\\
T_y
\end{pmatrix}
$$


$$\overline{T}_{ms}(f_{(t)}) =
\begin{pmatrix}
T_x\\
T_y
\end{pmatrix} =\frac12
\begin{pmatrix}
\mathcal{L}_u\left\{f(t) + f(-t)\right\}\\[0.2 cm]
\mathcal{L}_u\left\{f(t) -f(-t)\right\}
\end{pmatrix}
$$


Para reconstruir $f_{(t)}$ se debe resolver el sistema de ecuaciones:
$$
\begin{cases}
T_x =\frac12 \mathcal{L}_u\left\{f(t) + f(-t)\right\}\\[0.2 cm]
T_y = \frac12 \mathcal{L}_u\left\{f(t) -f(-t)\right\}
\end{cases}
$$


$$
\begin{cases}
\mathcal{L}_u^{-1}\left\{2 T_x\right\} =(f(t) + f(-t)) u(t)\\[0.2 cm]
\mathcal{L}_u^{-1}\left\{2 T_y\right\} =  (f(t) -f(-t)) u(t)
\end{cases}
$$

$$
\begin{cases}
2 \mathcal{L}^{-1}\left\{ T_x\right\}_{(t)} u(t)=f(t)u(t) + f(-t)u(t)\\[0.2 cm]
2 \mathcal{L}^{-1}\left\{T_y\right\}_{(t)} u(t)=f(t)u(t) -f(-t)u(t)
\end{cases}
$$

En este paso debemos tomar dos caminos, uno para recomponer la función en el inervalo donde $t\ge 0$, y otro para cuando $t<0$. Empezamos reconstruyendo el camino donde $t \ge 0$:

Despejando $f_{(-t)}$ de ambas ecuaciones, para después igualarlas tenemos que:

$$
\begin{cases}
f(-t)u(t) = 2 \mathcal{L}^{-1}\left\{ T_x\right\}_{(t)} u(t) - f(t)u(t)\\
f(-t)u(t) = -\left[2 \mathcal{L}^{-1}\left\{T_y\right\}_{(t)} u(t) - f(t)u(t)\right]
\end{cases}
$$

Igualando las ecuaciones nos queda:

$$
-\left[2 \mathcal{L}^{-1}\left\{T_y\right\}_{(t)} u(t) - f(t)u(t)\right] = 2 \mathcal{L}^{-1}\left\{ T_x\right\}_{(t)} u(t) - f(t)u(t)
$$

$$
-2 \mathcal{L}^{-1}\left\{T_y\right\}_{(t)} u(t) + f(t)u(t) = 2 \mathcal{L}^{-1}\left\{ T_x\right\}_{(t)} u(t) - f(t)u(t)  
$$

$$
2 f(t)u(t) = 2 \mathcal{L}^{-1}\left\{ T_x\right\}_{(t)} u(t) + 2 \mathcal{L}^{-1}\left\{T_y\right\}_{(t)} u(t) 
$$

\begin{equation}
f(t)u(t) = \left[ \mathcal{L}^{-1}\left\{ T_x\right\}_{(t)}+ \mathcal{L}^{-1}\left\{T_y\right\}_{(t)}\right]  u(t)
\label{parte_positiva}
\end{equation}

Similarmete acomo resolvimos la parte donde $t \ge 0$, resolvemos ahora para cuando $ t < 0$:

$$
\begin{cases}
f(t)u(t) = 2 \mathcal{L}^{-1}\left\{T_x\right\}_{(t)} u(t) - f(-t)u(t)\\
f(t)u(t) = 2 \mathcal{L}^{-1}\left\{T_y\right\}_{(t)} u(t) + f(-t)u(t)
\end{cases}
$$

$$
2 \mathcal{L}^{-1}\left\{T_y\right\}_{(t)} u(t) + f(-t)u(t) = 2 \mathcal{L}^{-1}\left\{T_x\right\}_{(t)} u(t) - f(-t)u(t)
$$

$$
2 f(-t)u(t) = 2 \mathcal{L}^{-1}\left\{T_x\right\}_{(t)} u(t) + 2 \mathcal{L}^{-1}\left\{T_y\right\}_{(t)} u(t)
$$

$$
 f(-t)u(t) =  \left[\mathcal{L}^{-1}\left\{T_x\right\}_{(t)}  - \mathcal{L}^{-1}\left\{T_y\right\}_{(t)} \right] u(t)
$$

Hacemos un cambio de variable para poder expresar a la función como $f_{(t)}$, lo cual nos quedaría como:
\begin{equation}
 f(t)u(-t) =  \left[\mathcal{L}^{-1}\left\{T_x\right\}_{(-t)}  - \mathcal{L}^{-1}\left\{T_y\right\}_{(-t)} \right] u(-t)
\label{parte_negativa}
\end{equation}

Tomando ahora las expresiones \ref{parte_positiva} y \ref{parte_negativa}, juntas nos definen la función original de la siguiente manera:

\begin{equation}
 f(t) = \left[ \mathcal{L}^{-1}\left\{ T_x\right\}_{(t)}+ \mathcal{L}^{-1}\left\{T_y\right\}_{(t)}\right]  u(t) +  \left[\mathcal{L}^{-1}\left\{T_x\right\}_{(-t)}  - \mathcal{L}^{-1}\left\{T_y\right\}_{(-t)} \right] u(-t)
\label{antitransformada}
\end{equation}


\section{Propiedades de la nueva transformada}


Se demuestran algunas propiedades de la nueva transformada, tomando como base las propiedades de la transformada de Laplace:


\subsection{Linealidad}

\begin{equation}
 \overline{T}_{ms}(a f_{(t)} + b g_{(t)}) = a \overline{T}_{ms}(f_{(t)}) + b  \overline{T}_{ms}(g_{(t)})
\label{linealidad}
\end{equation}

{\bfseries Demostración:}
{\small
$$
\overline{T}_{ms}(a f_{(t)} + b g_{(t)}) = \frac12
\begin{pmatrix}
\mathcal{L}\left\{[a f(t) + b g(t)]u(t)\right\}_{(s)} +  \mathcal{L}\left\{[a f(t) + b g(t)]u(-t)\right\}_{(-s)}\\[0.2 cm]
\mathcal{L}\left\{[a f(t) + b g(t)]u(t)\right\}_{(s)} -  \mathcal{L}\left\{[a f(t) + b g(t)]u(-t)\right\}_{(-s)}
\end{pmatrix}
$$

$$
\overline{T}_{ms}(a f_{(t)} + b g_{(t)}) = \frac12
\begin{pmatrix}
\mathcal{L}\left\{[a f(t)u(t)] + [b g(t)u(t)]\right\}_{(s)} +  \mathcal{L}\left\{[a f(t) u(-t)] + [b g(t)u(-t)]\right\}_{(-s)}\\[0.2 cm]
\mathcal{L}\left\{[a f(t)u(t)] + [b g(t)u(t)]\right\}_{(s)} -  \mathcal{L}\left\{[a f(t) u(-t)] + [b g(t)u(-t)]\right\}_{(-s)}
\end{pmatrix}
$$

$$
\overline{T}_{ms}(a f_{(t)} + b g_{(t)}) = \frac12
\begin{pmatrix}
\mathcal{L}\left\{[a f(t)u(t)]\right\} + \mathcal{L}\left\{[b g(t)u(t)]\right\}_{(s)} +  \mathcal{L}\left\{[a f(t) u(-t)]\right\} + \mathcal{L}\left\{[b g(t)u(-t)]\right\}_{(-s)}\\[0.2 cm]
\mathcal{L}\left\{[a f(t)u(t)]\right\} + \mathcal{L}\left\{[b g(t)u(t)]\right\}_{(s)} -  \mathcal{L}\left\{[a f(t) u(-t)]\right\} - \mathcal{L}\left\{[b g(t)u(-t)]\right\}_{(-s)}
\end{pmatrix}
$$

$$
\overline{T}_{ms}(a f_{(t)} + b g_{(t)}) = \frac12
\begin{pmatrix}
a\mathcal{L}\left\{[ f(t)u(t)]\right\} +  a\mathcal{L}\left\{[ f(t) u(-t)]\right\} + b\mathcal{L}\left\{[ g(t)u(t)]\right\}_{(s)}  + b\mathcal{L}\left\{[ g(t)u(-t)]\right\}_{(-s)}\\[0.2 cm]
a\mathcal{L}\left\{[f(t)u(t)]\right\} -  a\mathcal{L}\left\{[a f(t) u(-t)]\right\} + b\mathcal{L}\left\{[ g(t)u(t)]\right\}_{(s)}  - b\mathcal{L}\left\{[ g(t)u(-t)]\right\}_{(-s)}
\end{pmatrix}
$$


$$
\overline{T}_{ms}(a f_{(t)} + b g_{(t)}) = a \frac12
\begin{pmatrix}
\mathcal{L}\left\{[ f(t)u(t)]\right\} +  \mathcal{L}\left\{[ f(t) u(-t)]\right\}\\[0.2 cm]
\mathcal{L}\left\{[f(t)u(t)]\right\} -  \mathcal{L}\left\{[a f(t) u(-t)]\right\}
\end{pmatrix}
+b \frac12
\begin{pmatrix}
  \mathcal{L}\left\{[ g(t)u(t)]\right\}_{(s)}  + \mathcal{L}\left\{[ g(t)u(-t)]\right\}_{(-s)}\\[0.2 cm]
  \mathcal{L}\left\{[ g(t)u(t)]\right\}_{(s)}  - \mathcal{L}\left\{[ g(t)u(-t)]\right\}_{(-s)}
\end{pmatrix}
$$
}

$$
\therefore \overline{T}_{ms}(a f_{(t)} + b g_{(t)}) = a \overline{T}_{ms}(f_{(t)}) + b  \overline{T}_{ms}(g_{(t)})
$$

\subsection{Desplazamineto temporal}

\begin{equation}
 \overline{T}_{ms}(f_{(t-a)}) = \overline{T}_{ms}(f_{(t)}) e^{-at}
\label{desplazamiento_temporal}
\end{equation}

{\bfseries Demostración:}
$$\overline{T}_{ms}(f_{(t-a)}) = \frac12
\begin{pmatrix}
\mathcal{L}\left\{f(t-a)u(t-a)\right\}_{(s)} +  \mathcal{L}\left\{f(t-a)u(-(t-a))\right\}_{(-s)}\\[0.2 cm]
\mathcal{L}\left\{f(t-a)u(t-a)\right\}_{(s)} -  \mathcal{L}\left\{f(t-a)u(-(t-a))\right\}_{(-s)}
\end{pmatrix}$$


$$
\overline{T}_{ms}(f_{(t-a)}) = \frac12
\begin{pmatrix}
\mathcal{L}\left\{f(t)u(t)\right\}_{(s)} e^{-at} +  \mathcal{L}\left\{f(t)u(-t)\right\}_{(-s)} e^{-at} \\[0.2 cm]
\mathcal{L}\left\{f(t)u(t)\right\}_{(s)}  e^{-at} -  \mathcal{L}\left\{f(t)u(-t)\right\}_{(-s)} e^{-at} 
\end{pmatrix}
$$

$$
\overline{T}_{ms}(f_{(t-a)}) = \frac12
\begin{pmatrix}
\left[\mathcal{L}\left\{f(t)u(t)\right\}_{(s)} +  \mathcal{L}\left\{f(t)u(-t)\right\}_{(-s)}\right]e^{-at} \\[0.2 cm]
\left[\mathcal{L}\left\{f(t)u(t)\right\}_{(s)} -  \mathcal{L}\left\{f(t)u(-t)\right\}_{(-s)}\right]e^{-at} 
\end{pmatrix}
$$


$$
\overline{T}_{ms}(f_{(t-a)}) = \frac12
\begin{pmatrix}
\mathcal{L}\left\{f(t)u(t)\right\}_{(s)} +  \mathcal{L}\left\{f(t)u(-t)\right\}_{(-s)}\\[0.2 cm]
\mathcal{L}\left\{f(t)u(t)\right\}_{(s)} -  \mathcal{L}\left\{f(t)u(-t)\right\}_{(-s)}
\end{pmatrix} e^{-at}
$$

$$
\therefore \overline{T}_{ms}(f_{(t-a)}) = \overline{T}_{ms}(f_{(t)}) e^{-at}
$$

\subsection{Convolución}

Esta propiedad sólo de mantiene para funciones $f$ y $g$ tal que  \makebox{$f(t) = 0 \ \forall t < 0$} y  \makebox{$g(t) = 0 \ \forall t < 0$}. Debido a estas restricciones podemos expresar nuestra tranformada en función de la tranformada unilteral de Laplace.

$$
\overline{T}_{ms}(f_{(t)} * g_{(t)}) = \frac12
\begin{pmatrix}
\mathcal{L}_u\left\{(f_{(t)} * g_{(t)}) + (f_{(-t)} * g_{(-t)})\right\}\\[0.2 cm]
\mathcal{L}_u\left\{(f_{(t)} * g_{(t)}) - (f_{(-t)} * g_{(-t)})\right\}
\end{pmatrix}
$$

Sea $f(t)$ par, dado que $f(t) = f(-t)$, se puede escribir la transformada de la siguiente manera:

$$
\overline{T}_{ms}(f_{(t)} * g_{(t)}) = \frac12
\begin{pmatrix}
\mathcal{L}_u\left\{(f_{(t)} * g_{(t)}) + (f_{(t)} * g_{(-t)})\right\}\\[0.2 cm]
\mathcal{L}_u\left\{(f_{(t)} * g_{(t)}) - (f_{(t)} * g_{(-t)})\right\}
\end{pmatrix}
$$

Por propiedad de la convolución se sabe que $f * (g +h) = (f*g) + (f*h)$, por lo que se puede reducir la expresión de la transformada aplicando esta propiedad.
$$
\overline{T}_{ms}(f_{(t)} * g_{(t)}) = \frac12
\begin{pmatrix}
\mathcal{L}_u\left\{(f_{(t)} * (g_{(t)}+ g_{(-t)})\right\}\\[0.2 cm]
\mathcal{L}_u\left\{(f_{(t)} * ( g_{(t)} - g_{(-t)})\right\}
\end{pmatrix}
$$

$$
\overline{T}_{ms}(f_{(t)} * g_{(t)}) = \frac12
\begin{pmatrix}
\mathcal{L}_u\left\{(f_{(t)} \right\} \mathcal{L}_u\left\{ g_{(t)}+ g_{(-t)}\right\}\\[0.2 cm]
\mathcal{L}_u\left\{(f_{(t)} \right\} \mathcal{L}_u\left\{ g_{(t)} - g_{(-t)}\right\}
\end{pmatrix}
$$

$$
\overline{T}_{ms}(f_{(t)} * g_{(t)}) = \frac12
\begin{pmatrix}
 \mathcal{L}_u\left\{ g_{(t)}+ g_{(-t)}\right\}\\[0.2 cm]
\mathcal{L}_u\left\{ g_{(t)} - g_{(-t)}\right\}
\end{pmatrix} \mathcal{L}_u\left\{(f_{(t)} \right\}  =\overline{T}_{ms}( g_{(t)})  \mathcal{L}_u\left\{(f_{(t)} \right\}
$$

$$
\therefore \overline{T}_{ms}(f_{(t)} * g_{(t)}) =\overline{T}_{ms}( g_{(t)})  \mathcal{L}_u\left\{(f_{(t)} \right\}
$$


\subsection{Derivada temporal}

\begin{equation}\overline{T}_{ms}\left(\frac{d^n f_{(t)}}{dt^n}\right) =s^n\  \overline{T}_{ms}(f_{(t)})
- \sum\limits_{k=0}^{k=n-1} s^k
\begin{pmatrix}
 (f_{(t)} + f_{(-t)})^{(n-k-1)}_{(0^-)}\\[0.5 cm]
 (f_{(t)} - f_{(-t)})^{(n-k-1)}_{(0^-)}
\end{pmatrix}
\end{equation}

En el caso que la función cumpla que $f_{(t)} = 0 \ \forall t < 0$, la expresión quedaría:

$$\overline{T}_{ms}\left(\frac{d^n f_{(t)}}{dt^n}\right) = \frac12 s^n\mathcal{L}_u\left\{f_{(t)}\right\} -\sum\limits_{k=0}^{k=n-1} s^k f^{(n-k-1)}_{(0^-)}
\begin{pmatrix}
1\\
1
\end{pmatrix}
$$

{\bfseries Demostración:}

$$\overline{T}_{ms}\left(\frac{d^n f_{(t)}}{dt^n}\right) = \frac12
\begin{pmatrix}
\mathcal{L}_u\left\{\frac{d^n f_{(t)}}{dt^n} + \frac{d^n f_{(-t)}}{dt^n}\right\}\\[0.2 cm]
\mathcal{L}_u\left\{\frac{d^n f_{(t)}}{dt^n} - \frac{d^n f_{(-t)}}{dt^n}\right\}
\end{pmatrix}
$$

$$\overline{T}_{ms}\left(\frac{d^n f_{(t)}}{dt^n}\right) = \frac12
\begin{pmatrix}
\mathcal{L}_u\left\{\frac{d^n (f_{(t)} +f_{(-t)})}{dt^n}\right\}\\[0.2 cm]
\mathcal{L}_u\left\{\frac{d^n (f_{(t)}- f_{(-t)})}{dt^n}\right\}
\end{pmatrix}
$$

$$\overline{T}_{ms}\left(\frac{d^n f_{(t)}}{dt^n}\right) = \frac12
\begin{pmatrix}
s^n\mathcal{L}_u\left\{(f_{(t)} + f_{(-t)})\right\} -\sum\limits_{k=0}^{k=n-1} s^k (f_{(t)} + f_{(-t)})^{(n-k-1)}_{(0^-)}\\[0.5 cm]
s^n\mathcal{L}_u\left\{(f_{(t)} - f_{(-t)})\right\} -\sum\limits_{k=0}^{k=n-1} s^k (f_{(t)} - f_{(-t)})^{(n-k-1)}_{(0^-)}
\end{pmatrix}
$$

$$\overline{T}_{ms}\left(\frac{d^n f_{(t)}}{dt^n}\right) =s^n \frac12
\begin{pmatrix}
\mathcal{L}_u\left\{(f_{(t)} + f_{(-t)})\right\} \\[0.5 cm]
\mathcal{L}_u\left\{(f_{(t)} - f_{(-t)})\right\} 
\end{pmatrix}
- \sum\limits_{k=0}^{k=n-1} s^k
\begin{pmatrix}
 (f_{(t)} + f_{(-t)})^{(n-k-1)}_{(0^-)}\\[0.5 cm]
 (f_{(t)} - f_{(-t)})^{(n-k-1)}_{(0^-)}
\end{pmatrix}
$$

$$\overline{T}_{ms}\left(\frac{d^n f_{(t)}}{dt^n}\right) =s^n\  \overline{T}_{ms}(f_{(t)})
- \sum\limits_{k=0}^{k=n-1} s^k
\begin{pmatrix}
 (f_{(t)} + f_{(-t)})^{(n-k-1)}_{(0^-)}\\[0.5 cm]
 (f_{(t)} - f_{(-t)})^{(n-k-1)}_{(0^-)}
\end{pmatrix}
$$

\section{Tabla de transformada de funciones elementales}







\end{document} 